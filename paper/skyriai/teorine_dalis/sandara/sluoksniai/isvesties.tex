\subsubsection{Išvesties sluoksnis}
\label{skr:dnt_isvesties_sluoksnis}

Išvesties sluoksnis dirbtiniuose neuroniniuose tinkluose yra naudojamas modelio rezultatams generuoti. Neuronų kiekis šiame sluoksnyje priklauso nuo sprendžiamos užduoties tipo. Klasifikavimo uždaviniuose išvesties sluoksnio neuronų skaičius paprastai sutampa su galimų klasių skaičiumi, o kiekvieno neurono išvestis interpretuojama kaip tikimybė, kad įvesties duomenys priklauso tai klasei. Tuo tarpu regresijos uždaviniuose dažniausiai naudojamas vienas neuronas, kurio išvestis atitinka prognozuojamą reikšmę.

