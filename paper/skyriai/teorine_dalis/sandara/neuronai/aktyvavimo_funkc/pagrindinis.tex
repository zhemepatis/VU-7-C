\subsubsection{Aktyvavimo funkcijos}
\label{skr:aktyvavimo_funkc}

Dirbtiniai neuroniniai tinklai taikomi ne tik su tiesiniais, bet ir su netiesines savybes turinčiais duomenim. Jei modeliai naudotų vien svertinės sumos funkcijos apskaičiavimus, jie generuotų tik tiesines išvestis ir nebūtų tinkami atpažinti sudėtingesnius duomenų raštus. Siekiant įveikti šį apribojimą, neuroniniuose tinkluose yra naudojamos aktyvavimo funkcijos.

Aktyvavimo funkcijos pasirinkimas labai priklauso nuo uždavinio specifikos. Žemiau esančiuose poskyriuose \ref{skr:sigmoidine}, \ref{skr:hiperbolinio_tangento}, \ref{skr:relu} yra pateikiamos dažniausiai dirbtiniuose neuroniniuose tinkluose naudojamos aktyvavimo funkcijos.

\subsubsubsection{Sigmoidinė funkcija}
\label{skr:sigmoidine}

    \[
        \sigma(x) = \frac{1}{1 + e^{-x}}
    \]
\subsubsubsection{Hiperbolinio tangento funkcija}
\label{skr:hiperbolinio_tangento}

\[
    f(x) = \tanh(x) = \frac{e^x - e^-x}{e^x + e^-x}
\]
\subsubsubsection{Rektifikuoto tiesinio elemento funkcija}
\label{skr:relu}

\[
    \text{ReLU}(x) =
    \begin{cases}
        0, \text{ jei $x \leq 0$ }  \\ 
        x, \text{ jei $x > 0$ }  \\ 
    \end{cases}
\]