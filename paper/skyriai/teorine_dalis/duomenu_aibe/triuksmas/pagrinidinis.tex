\subsection{Triukšmo duomenų aibei pritaikymas}

Dirbant su realaus pasaulio duomenimis dažnai susiduriama su tuo, kad jie nėra visiškai tikslūs. To priežastys gali būti įvairios - žmogaus klaidos, prietaisų tikslumo ribos, aplinkos veiksnių poveikis ir t.t. Tokie nedideli netikslumai duomenyse yra vadinimai triukšmu.

Kadangi realaus pasaulio sąlygomis surinkti duomenys retai būna tikslūs, mašininio mokymosi metodai turi gebėti gerai dirbti su duomenimis, kuriuose yra triukšmo. Dėl šios priežasties apmokant mašininio mokymosi modelius su dirbtinai generuotais duomenimis, taip pat yra duomenims pridėti tam tikrą triukšmo lygį.

\subsubsection{Gauso skirstinio metodas}

    Vienas iš triukšmo pridėjimo būdų pasitelkia Gauso skirstinį.

    Žemiau pateiktuose paveikslėliuose galime pamatyti, kaip $\sigma$ triukšmo parametro dydis lemia taškų išsidėstymą.

    Žemiau pateiktame paveikslėlyje matome triukšmo pritaikymo pavyzdį dvimatėje erdvėje. 

    \begin{figure}[H]
        \centering
        \includegraphics[width=0.8\linewidth]{attachments/img/triuksmo_pvz_0.png}
        \caption{Švarių ir duomenų su triukšmu sugretinimas, kai ...}
        \label{pav:triuksmo_ir_svariu_duomenu_sugretinimas_0}
    \end{figure}

    \begin{figure}[H]
        \centering
        \includegraphics[width=0.8\linewidth]{attachments/img/triuksmo_pvz_1.png}
        \caption{Švarių ir duomenų su triukšmu sugretinimas, kai ...}
        \label{pav:triuksmo_ir_svariu_duomenu_sugretinimas_1}
    \end{figure}

    Šio kursinio kontekste paklaidos buvo generuojamas pasitelkiant normalųjį skirstinį, kur paklaidos vidurkis yra 0, o standartinis nuokrypis 0.5.