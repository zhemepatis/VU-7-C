\subsection{Dirbtinis duomenų generavimas}

Dirbtinis duomenų generavimas - tai duomenų kūrimas, naudojant programinius algoritmus, siekiant atkartoti realaus pasaulio duomenų statistines savybes. 

Dirbtinai sugeneruoti duomenys leidžia užtikrinti didelį duomenų kiekį bei jų įvairumą. Dėl šios priežasties šis duomenų kūrimo būdas yra plačiai naudojamas tais atvejais, kai realaus pasaulio duomenų yra per mažai, arba kai jie yra sunkiai prieinami dėl finansinių priežasčių. 

Be to dirbtinis duomenų generavimas yra naudingas situacijose, kai dirbama su duomenimis, reguliuojamais asmens duomenų apsaugos įstatymais. Tokiuose atvejuose sintetiniai duomenys leidžia išvengti jautrių duomenų nutekėjimo, kadangi modelių apmokymui nėra naudojami su konkrečiais asmenimis susiję duomenys. 

Kitas scenarijus, aktualus ir šio kursinio kontekste, yra kai duomenys neprivalo turėti kažkokio konkretaus domeno. Pavyzdžiui, tiriant mašininio mokymosi metodų savybes yra pravartu modelius išbandyti su skirtingo pobūdžio duomenimis. Tad dirbtinis duomenų generavimas suteikia galimybę lengvai sugeneruoti didelį kiekį įvairias savybes turinčių duomenų.

Šiame darbe duomenys buvo generuojami dirbtinai, pasitelkiant sferos, ... etalono funkcijas (angl. \enquote{benchmark functions}). Jos yra aprašomos atitinkamai \ref{skr:sferos_funkcija}, \ref{} bei \ref{} poskyriuose.

\subsubsection{Sferos funkcija}
\label{skr:sferos_funkcija}

Sferos funkcija yra viena iš paprasčiausių etalono funkcijų, kurios pavidalą galima rasti žemiau pateiktoje \ref{lgt:sferos_funkcija} formulėje.

\begin{equation}
    f(X) = \sum_{i = 0}^{n}{x_i^2}
    \label{lgt:sferos_funkcija}
\end{equation}

Čia $x_i$ yra vektoriaus $X$ komponentės, o $n$ duomenų dimensijos dydis. 

Šios funkcijos grafikas yra simetriškas ir pasižymi išgaubtumu. Žemiau pateiktas \ref{pav:sferos_funkcijos_grafikas} paveikslėlis atvaizduoja sferos funkcijos grafiką dvimačiu atveju.

\begin{figure}[H]
    \centering
    \includegraphics[width=0.7\linewidth]{img/sferos_funkc_grafikas.png}
    \caption{Sferos funkcijos generuojamų duomenų grafikas}
    \label{pav:sferos_funkcijos_grafikas}
\end{figure}