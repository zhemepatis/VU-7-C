\section{Praktinė dalis}

    Siekiant įvertinti, kaip dirbtiniai neuroniniai tinklai reaguoja į triukšmą duomenyse, buvo atlikti bandymai su sukurtu dirbtinio neuroninio tinklo modeliu (\ref{} skyrius). Papildomai, norint įvertinti, kiek geriau dirbtinis neuroninis tinklas susitvarko su triukšmu duomenyse, tam pačiam uždaviniui buvo pritaikyti ir keli k artimiausių kaimynų modeliai (\ref{} skyrius). Visi modeliai buvo pritaikyti tiek su švariom duomenų aibėm, tiek su aibėm, kuriose yra pritaikytas tam tikras triukšmo lygis.
        
    % TODO: move somwhere else
    % Prieš pereinant prie vėlesniuose skyreliuose aptariamų tyrimų ir juose rastų rezultatų, pravartu aptarti juose vartojamas sąvokas. Šio darbo kontekste laikysime, kad \enquote{epocha} tai yra modelio perėjimas per visą apmokymo duomenų aibę, o \enquote{iteracija} yra modelio svertų atnaujinimas.

    \subsection{Duomenų aibės kūrimas}

        Bandymams atlikti buvo generuojami keturmačiai duomenys, kurių kiekviena komponentė buvo parenkama iš intervalo $[-5; 5]$. Duomenų generavimui buvo pasitelktos sferos (\ref{eq:sferos_etalono_funkc} lygtis), ... (\ref{} lygtis) bei ... (\ref{} lygtis) etalono funkcijos. 
        
        Mašininio mokymosi modeliai buvo pritaikyti skirtingo dydžio kiekvienos rūšies duomenų aibėms. Dirbtinis neuroninis tinklas buvo pritaikytas su 1~000, 10~000, 100~000, 1~000~000 dydžio duomenų aibėmis, tuo tarpu k artimiausių kaimynų metodas dar papildomai buvo išbandytas su aibe, turinčia 10~000~000 įrašų. 
        
        Triukšmo pridėjimui į duomenų aibę buvo pasitelktas Gauso paskirsymo funkcijos metodas (\ref{} skyrius). Paklaidos buvo pridedamos prie sugeneruotų etalono funkcijų reikšmių. Triukšmas buvo generuojamos su parametrais $\mu = 0$, $\sigma = 0.5$.

        Taip pat verta paminėti, kad prieš apmokant modelius duomenys dar papildomai buvo normalizuojami pasitelkiant min-max metodą. Šis metodas leidžia skaitines reikšmes sutraukti į intervalą $[0; 1]$ taip išvengiant skirtingų požymių mastelių įtakos mokymo procesui. Tai yra atliekama pasitelkiant formulę

        \begin{equation}
            \label{eq:min_max_funkc}
            x' = \frac{x - \min(x)}{\max(x) - \min(x)}
        \end{equation}

    \subsection{Bandymai su dirbtiniu neuroniniu tinklu}

    \subsubsection{Modelis}

        Siekiant panagrinėti, kaip dirbtinis neuroninis tinklas susitvarko su triukšmingais duomenimis, buvo sukurtas modelis, sudarytas iš vieno įvesties sluoksnio (4 neuronai), vieno paslėpto sluoksnio (70 neuronų) bei vieno išvesties sluoksnio (1 neuronas). Paslėptajame sluoksnyje buvo panaudota sigmoidinė aktyvavimo funkcija, tuo tarpu išvesties sluoksnyje aktyvavimo funkcija pritaikyta nebuvo. Pastarasis sprendimas buvo priimtas dėl priežasties, kad dirbtinis neuroninis tinklas buvo pritaikytas spręsti regresijos uždavinį.

        Bandymų metu buvo pastebėta, kad geriausius rezultatus validacijos metu dirbtinio neuroninio tinklo modelis grąžina, kai mokymosi greitis yra $0,01$, o mokymosi bei validavoimo aibės yra padalintos į dalis po 8 įrašus.

        Modelio apmokymo etapo ilgis buvo stabdomas įvykus vienai iš stabdymo sąlygų:
        \begin{itemize}
            \item klaidos funkcijos reikšmė $13$ epochų iš eilės mažėja mažiau nei per $10^{-6}$;
            \item buvo pasiektas maksimalus $150$ epochų kiekis.
        \end{itemize}

        Modelio apmokymui buvo skiriama $70\%$ aibės duomenų, $15\%$ buvo skiriama modelio validavimui, ir kiti $15\%$ - modelio testavimui. 

    \subsubsection{Rezultatai su sferos funkcija}
        
        Žemiau pateiktose \ref{tbl:fnn_sferos_funkc_svarus} ir \ref{tbl:fnn_sferos_funkc_su_triuksmu} lentelėse yra matomi dirbtinio neuroninio tinklo rezultatai su sferos funkcijos generuotais duomenimis. Verta paminėti, jog prieš skaičiuojant absoliutinės paklaidos rodiklius, modelio spėjimai buvo konvertuoti į originalų duomenų mastelį. Taigi atsižvelgus, kad šio bandymo metu duomenys buvo apibrėžiami sferos funkcija, maksimali galima absoliutinė paklaida šiuo atveju yra 100.

        Nagrinėjant rezultatus su švariais duomenim

        \begin{table}[H]
    \centering
    
    \begin{tabular}{|c|c|c|c|c|}
        \hline
        \textbf{Taškų kiekis} & \textbf{Minimumas} & \textbf{Maksimumas} & \textbf{Vidurkis} & \textbf{Standartinis nuokrypis} \\ \hline
        1 000      & 0.002857 & 6.881401 & 1.773480 & 1.231157 \\ \hline
        10 000     & 0.001137 & 4.690506 & 0.625588 & 0.514914 \\ \hline
        100 000    & 0.000001 & 1.170944 & 0.217896 & 0.172228 \\ \hline
        1 000 000  & 0.000004 & 1.631981 & 0.416016 & 0.274041 \\ \hline
    \end{tabular}
    
    \caption{Dirbtinio neuroninio tinklo rezultatai su švariais sferos funkcijos generuotais duomenis}
    \label{tbl:fnn_sferos_funkc_svarus}
\end{table}

        \begin{table}[H]
    \centering
    
    \begin{tabular}{|c|c|c|c|c|}
        \hline
        \textbf{Taškų kiekis} & \textbf{Minimumas} & \textbf{Maksimumas} & \textbf{Vidurkis} & \textbf{Standartinis nuokrypis} \\ \hline
        1 000      & 0.008972 & 9.575722 & 1.970832 & 1.910026 \\ \hline
        10 000     & 0.000362 & 5.672182 & 2.127527 & 1.227688 \\ \hline
        100 000    & 0.000038 & 2.539330 & 0.536711 & 0.420499 \\ \hline
        1 000 000  & 0.000006 & 1.809952 & 0.419722 & 0.278947 \\ \hline
    \end{tabular}
    
    \caption{Dirbtinio neuroninio tinklo rezultatai su triukšmingais duomenim}
    \label{tbl:fnn_sferos_funkc_su_triuksmu}
\end{table}


        Nagrinėjant gautus rezultatus lentelėse, galima pastebėti, jog ...  



% \subsection{Dirbtinio neuroninio tinklo rezultatai}


% \subsubsection{Sferos funkcija}

%     \subsubsubsection{Švarūs duomenys}


%         Matome, kad didinant taškų kiekį nuo 1~000 iki 100~000, modelio daromų paklaidų vidurkis ir standartinis nuokrypis mažėjo, taigi galime teigti, kad modelio tikslumas didėjo. Tačiau verta atkreipti dėmesį, kad padidinus taškų aibės dydį iki 1~000~000 modelio paklaidų statistiniai rodikliai pablogėjo.

%     \subsubsubsection{Triukšmingi duomenys}

%         Šiame bandyme buvo tiriamas neuroninio tinklo gebėjimas interpoliuoti duomenis, kurie pasižymi sferos funkcijos (\ref{lgt:sferos_funkcija} lygtis) savybėmis jiems pridėjus triukšmo.


%         Iš \ref{tbl:fnn_sferos_funkc_su_triuksmu} lentelės matoma, kad bendru atveju vidurkis didinant taškų kiekį mažėja. Atskiras atvejis yra, kai taškų kiekis buvo padidintas iki 10~000 - tuo atveju absoliutinės paklaidos vidurkis padidėjo.

%     \subsubsubsection{Palyginimas}

%         Matome, kad modelio absoliutinės paklaidos vidurkis arba ženkliai nesikeitė, arba paklaidos vidurkio padidėjimas svyravo nuo 1.1 iki 3.4 kartų. Standartinio nuokrypis taip pat arba nerodė didelių pakitimų, arba paklaidos standartinio nuokrypio padidėjimas svyravo nuo 1.6 iki 2.4.

    \subsection{K artimiausių kaimynų metodo rezultatai}

        Tyrimas buvo atliktas su 1~000, 10~000, 100~000, 1~000~000 ir 10~000~000 taškų aibėmis, kurių $85\%$ buvo skiriami modelio apmokymui ir validavimui, likę $15\%$ - modelio testavimui.

        Bandymui buvo generuojami keturmačiai duomenys ($n = 4$), kurių kiekviena komponentė buvo atsitiktinai parenkama iš intervalo $[-5; 5]$. Prieš apmokant modelį, duomenys buvo papildomai apdorojami: kiekviena vektoriaus komponentė bei sferos funkcijos reikšmės buvo normalizuojamos naudojant min–max metodą.

        Modelis buvo validuojamas pasitelkiant kryžminę validaciją (angl. \textit{cross-validation}). Bandymų metu buvo nustatyta, kad geriausiai modelis veikia, kai kaimynų skaičius yra lygus $3$, tačiau palyginimui buvo atlikti bandymai ir kai artimiausias kaimynas yra vienas.

        Visais atvejais prieš skaičiuojant statistinius absoliutinės paklaidos rodiklius, modelio spėjimai buvo konvertuoti į originalų duomenų mastelį. Taigi atsižvelgus, kad tyrimo metu duomenys buvo apibrėžiami sferos funkcija, maksimali galima absoliutinė paklaida šio tyrimo atveju yra 100.

        \subsubsection{Sferos funkcija, kai k = 1}

            \subsubsubsection{Švarūs duomenys}

    Šiame bandyme buvo tiriamas K artimiausių kaimynų metodo gebėjimas interpoliuoti duomenis, kurie pasižymi sferos funkcijos savybėmis. Šiuo atveju $k = 1$.

    \begin{table}[H]
    \centering
    
    \begin{tabular}{|c|c|c|c|c|}
        \hline
        \textbf{Taškų kiekis} & \textbf{Minimumas} & \textbf{Maksimumas} & \textbf{Vidurkis} & \textbf{Standartinis nuokrypis} \\ \hline
        1 000      & 0.050320 & 20.178803 & 5.182272 & 3.617230 \\ \hline
        10 000     & 0.002558 & 13.502786 & 2.975179 & 2.329911 \\ \hline
        100 000    & 0.000256 & 11.192788 & 1.698232 & 1.322762 \\ \hline
        1 000 000  & 0.000007 & 6.239792 & 0.959560 & 0.743995 \\ \hline
        10 000 000  & 0.00000049 & 3.839611 & 0.538008 & 0.413742 \\ \hline
    \end{tabular}
    
    \caption{K artimiausių kaimynų metodo absoliutinės paklaidos statistiniai rodikliai nagrinėjant sferos funkciją (k = 1)}
    \label{lnt:nn_sferos_funkc_rodikliai_0}
\end{table}

            \subsubsubsection{Triukšmingi duomenys}

Šiame bandyme buvo tiriamas K artimiausių kaimynų metodo gebėjimas interpoliuoti duomenis, kurie pasižymi sferos funkcijos savybėmis, jiems pridėjus triukšmo. Šiuo atveju $k = 1$.

\begin{table}[H]
    \centering
    
    \begin{tabular}{|c|c|c|c|c|}
        \hline
        \textbf{Taškų kiekis} & \textbf{Minimumas} & \textbf{Maksimumas} & \textbf{Vidurkis} & \textbf{Standartinis nuokrypis} \\ \hline
        1 000        & 0.028075 & 20.031243 & 5.220319 & 3.692268 \\ \hline
        10 000       & 0.007079 & 14.064025 & 2.997375 & 2.348190 \\ \hline
        100 000      & 0.000221 & 10.859588 & 1.743053 & 1.354251 \\ \hline
        1 000 000    & 0.000006 & 6.902217  & 1.042092 & 0.800102 \\ \hline
        10 000 000   & 0.000000 & 4.408415  & 0.670634 & 0.510044 \\ \hline
    \end{tabular}
    
    \caption{K artimiausių kaimynų metodo absoliutinės paklaidos statistiniai rodikliai nagrinėjant sferos funkciją (k = 1)}
    \label{lnt:nn_sferos_funkc_rodikliai_1}
\end{table}

            \subsubsubsection{Palyginimas}

Iš pateiktų \ref{lnt:nn_sferos_funkc_rodikliai_0} ir \ref{lnt:nn_sferos_funkc_rodikliai_1} lentelių matome, kad pridėjus triukšmą k artimiausių kaimynų interpeliacijos rezultatų tikslumas žymiai nesikeitė. Didžiausias absoliutinės paklaidos vidurkio padidėjimas siekia 1,2 karto. Tokias pačias įžvalgas galima padaryti ir nagrinėjant standartinį nuokrypį. 

        \subsubsection{Sferos funkcija, kai k = 3}

            \input{skyriai/praktine_dalis/knn/knn_rodikliai_0}
            \subsubsubsection{Triukšmingi duomenys}

    Šiame bandyme buvo tiriamas K artimiausių kaimynų metodo gebėjimas interpoliuoti duomenis, kurie pasižymi sferos funkcijos savybėmis. Šiuo atveju $k = 3$.

    \input{attachments/tables/knn_sferos_funkc_rodikliai_1}
            \subsubsubsection{Palyginimas}

    Iš aukščiau esančių \ref{tbl:knn_sferos_funkc_svarus} ir \ref{tbl:knn_sferos_funkc_su_triuksmu} lentelių galima pamatyti, kad taikant artimiausių kaimynų metodą, kai $k = 3$ tikslumas mažėja neženkliai - didžiausias absoliutinės paklaidos vidurkio padidėjimas siekia 1,2 karto. Standartinio nuokrypis padidėja iki 1,1 karto.

        \subsubsection{K artimiausių kaimynų metodo rezultatų palyginimas}

            Palyginus metodo rezultatus, kai $k = 1$ ir $k = 3$ yra akivaizdu, kad su didesniu kaimynų kiekiu, modelis grąžina geresnius rezultatus. Taip pat matome, kad abiem atvejais, modelių grąžinami rezultatai stabiliai gerėjo.
