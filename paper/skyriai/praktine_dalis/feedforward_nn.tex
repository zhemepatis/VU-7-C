\subsection{Dirbtinio neuroninio tinklo rezultatai}

    Tyrimas buvo atliktas su 1~000, 10~000, 100~000 ir 1~000~000 taškų aibėmis, kurių $70\%$ buvo skiriami modelio apmokymui, $15\%$ - modelio validavimui ir kiti $15\%$ - modelio testavimui.

    Bandymui buvo generuojami keturmačiai duomenys ($n = 4$), kurių kiekviena komponentė buvo atsitiktinai parenkama iš intervalo $[-5; 5]$. Prieš apmokant modelį, duomenys buvo papildomai apdorojami: kiekviena vektoriaus komponentė bei sferos funkcijos reikšmės buvo normalizuojamos naudojant min–max metodą.

    Pats neuroninio tinklo modelis buvo sudarytas iš vieno įvesties, vieno paslėpto ir vieno išvesties sluoksnių. Kiekvienas iš šių sluoksnių susidėjo iš atitinkamai 4, 70, ir 1 neuronų. Paslėptame sluoksnyje buvo taikoma zigmoidinė aktyvavimo funkcija, tuo tarpu išvesties sluoksnyje buvo palikta tiesinė aktyvavimo funkcija. Bandymo metu buvo pastebėta, kad geriausius rezultatus sukurtas modelis pasiekia, kai mokymosi greitis yra $0,01$. Modelis buvo apmokomas padalinus apmokymo taškų aibę į dalis po 8 taškus.

    Visais atvejais prieš skaičiuojant statistinius absoliutinės paklaidos rodiklius, modelio spėjimai buvo konvertuoti į originalų duomenų mastelį. Taigi atsižvelgus, kad tyrimo metu duomenys buvo apibrėžiami sferos funkcija, maksimali galima absoliutinė paklaida šio tyrimo atveju yra 100.

    Modelio apmokymo etapo ilgis buvo stabdomas įvykus vienai iš stabdymo sąlygų:
    \begin{itemize}
        \item klaidos funkcijos reikšmė $13$ epochų iš eilės mažėja mažiau nei per $10^{-6}$;
        \item buvo pasiektas maksimalus $150$ epochų kiekis.
    \end{itemize}


\subsubsection{Sferos funkcija}

    \subsubsubsection{Švarūs duomenys}

        Šiame bandyme buvo tiriamas neuroninio tinklo gebėjimas interpoliuoti švarius duomenis, kurie pasižymi sferos funkcijos (\ref{lgt:sferos_funkcija} lygtis) savybėmis.

        \input{attachments/tables/fnn_sferos_funkc_rodikliai_0}

        Matome, kad didinant taškų kiekį nuo 1~000 iki 100~000, modelio daromų paklaidų vidurkis ir standartinis nuokrypis mažėjo, taigi galime teigti, kad modelio tikslumas didėjo. Tačiau verta atkreipti dėmesį, kad padidinus taškų aibės dydį iki 1~000~000 modelio paklaidų statistiniai rodikliai pablogėjo.

    \subsubsubsection{Triukšmingi duomenys}

        Šiame bandyme buvo tiriamas neuroninio tinklo gebėjimas interpoliuoti duomenis, kurie pasižymi sferos funkcijos (\ref{lgt:sferos_funkcija} lygtis) savybėmis jiems pridėjus triukšmo.

        \input{attachments/tables/fnn_sferos_funkc_rodikliai_1}

        Iš \ref{lnt:fnn_sferos_funkc_rodikliai_1} lentelės matoma, kad bendru atveju vidurkis didinant taškų kiekį mažėja. Atskiras atvejis yra, kai taškų kiekis buvo padidintas iki 10~000 - tuo atveju absoliutinės paklaidos vidurkis padidėjo.

    \subsubsubsection{Palyginimas}

        Matome, kad modelio absoliutinės paklaidos vidurkis arba ženkliai nesikeitė, arba paklaidos vidurkio padidėjimas svyravo nuo 1.1 iki 3.4 kartų. Standartinio nuokrypis taip pat arba nerodė didelių pakitimų, arba paklaidos standartinio nuokrypio padidėjimas svyravo nuo 1.6 iki 2.4.