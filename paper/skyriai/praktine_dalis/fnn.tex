\subsection{Bandymai su dirbtiniu neuroniniu tinklu}

    \subsubsection{Modelis}

        Siekiant panagrinėti, kaip dirbtinis neuroninis tinklas susitvarko su triukšmingais duomenimis, buvo sukurtas modelis, sudarytas iš vieno įvesties sluoksnio (4 neuronai), vieno paslėpto sluoksnio (70 neuronų) bei vieno išvesties sluoksnio (1 neuronas). Paslėptajame sluoksnyje buvo panaudota sigmoidinė aktyvavimo funkcija, tuo tarpu išvesties sluoksnyje aktyvavimo funkcija pritaikyta nebuvo. Pastarasis sprendimas buvo priimtas dėl priežasties, kad dirbtinis neuroninis tinklas buvo pritaikytas spręsti regresijos uždavinį.

        Bandymų metu buvo pastebėta, kad geriausius rezultatus validacijos metu dirbtinio neuroninio tinklo modelis grąžina, kai mokymosi greitis yra $0,01$, o mokymosi bei validavoimo aibės yra padalintos į dalis po 8 įrašus.

        Modelio apmokymo etapo ilgis buvo stabdomas įvykus vienai iš stabdymo sąlygų:
        \begin{itemize}
            \item klaidos funkcijos reikšmė $13$ epochų iš eilės mažėja mažiau nei per $10^{-6}$;
            \item buvo pasiektas maksimalus $150$ epochų kiekis.
        \end{itemize}

        Modelio apmokymui buvo skiriama $70\%$ aibės duomenų, $15\%$ buvo skiriama modelio validavimui, ir kiti $15\%$ - modelio testavimui. 

    \subsubsection{Rezultatai su sferos funkcija}
        
        Žemiau pateiktose \ref{tbl:fnn_sferos_funkc_svarus} ir \ref{tbl:fnn_sferos_funkc_su_triuksmu} lentelėse yra matomi dirbtinio neuroninio tinklo rezultatai su sferos funkcijos generuotais duomenimis. Verta paminėti, jog prieš skaičiuojant absoliutinės paklaidos rodiklius, modelio spėjimai buvo konvertuoti į originalų duomenų mastelį. Taigi atsižvelgus, kad šio bandymo metu duomenys buvo apibrėžiami sferos funkcija, maksimali galima absoliutinė paklaida šiuo atveju yra 100.

        Nagrinėjant rezultatus su švariais duomenim

        \begin{table}[H]
    \centering
    
    \begin{tabular}{|c|c|c|c|c|}
        \hline
        \textbf{Taškų kiekis} & \textbf{Minimumas} & \textbf{Maksimumas} & \textbf{Vidurkis} & \textbf{Standartinis nuokrypis} \\ \hline
        1 000      & 0.002857 & 6.881401 & 1.773480 & 1.231157 \\ \hline
        10 000     & 0.001137 & 4.690506 & 0.625588 & 0.514914 \\ \hline
        100 000    & 0.000001 & 1.170944 & 0.217896 & 0.172228 \\ \hline
        1 000 000  & 0.000004 & 1.631981 & 0.416016 & 0.274041 \\ \hline
    \end{tabular}
    
    \caption{Dirbtinio neuroninio tinklo rezultatai su švariais sferos funkcijos generuotais duomenis}
    \label{tbl:fnn_sferos_funkc_svarus}
\end{table}

        \begin{table}[H]
    \centering
    
    \begin{tabular}{|c|c|c|c|c|}
        \hline
        \textbf{Taškų kiekis} & \textbf{Minimumas} & \textbf{Maksimumas} & \textbf{Vidurkis} & \textbf{Standartinis nuokrypis} \\ \hline
        1 000      & 0.008972 & 9.575722 & 1.970832 & 1.910026 \\ \hline
        10 000     & 0.000362 & 5.672182 & 2.127527 & 1.227688 \\ \hline
        100 000    & 0.000038 & 2.539330 & 0.536711 & 0.420499 \\ \hline
        1 000 000  & 0.000006 & 1.809952 & 0.419722 & 0.278947 \\ \hline
    \end{tabular}
    
    \caption{Dirbtinio neuroninio tinklo rezultatai su triukšmingais duomenim}
    \label{tbl:fnn_sferos_funkc_su_triuksmu}
\end{table}


        Nagrinėjant gautus rezultatus lentelėse, galima pastebėti, jog ...  



% \subsection{Dirbtinio neuroninio tinklo rezultatai}


% \subsubsection{Sferos funkcija}

%     \subsubsubsection{Švarūs duomenys}


%         Matome, kad didinant taškų kiekį nuo 1~000 iki 100~000, modelio daromų paklaidų vidurkis ir standartinis nuokrypis mažėjo, taigi galime teigti, kad modelio tikslumas didėjo. Tačiau verta atkreipti dėmesį, kad padidinus taškų aibės dydį iki 1~000~000 modelio paklaidų statistiniai rodikliai pablogėjo.

%     \subsubsubsection{Triukšmingi duomenys}

%         Šiame bandyme buvo tiriamas neuroninio tinklo gebėjimas interpoliuoti duomenis, kurie pasižymi sferos funkcijos (\ref{lgt:sferos_funkcija} lygtis) savybėmis jiems pridėjus triukšmo.


%         Iš \ref{tbl:fnn_sferos_funkc_su_triuksmu} lentelės matoma, kad bendru atveju vidurkis didinant taškų kiekį mažėja. Atskiras atvejis yra, kai taškų kiekis buvo padidintas iki 10~000 - tuo atveju absoliutinės paklaidos vidurkis padidėjo.

%     \subsubsubsection{Palyginimas}

%         Matome, kad modelio absoliutinės paklaidos vidurkis arba ženkliai nesikeitė, arba paklaidos vidurkio padidėjimas svyravo nuo 1.1 iki 3.4 kartų. Standartinio nuokrypis taip pat arba nerodė didelių pakitimų, arba paklaidos standartinio nuokrypio padidėjimas svyravo nuo 1.6 iki 2.4.