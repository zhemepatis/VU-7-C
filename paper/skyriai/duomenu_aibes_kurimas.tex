\section{Duomenų aibės kūrimas}

    \subsection{Dirbtinis duomenų generavimas}
            
        Dirbtinis duomenų generavimas -- tai duomenų kūrimo procesas, kurio metu, pasitelkiant programinius algoritmus, siekiama atkartoti realaus pasaulio duomenų statistines savybes. Kadangi šis duomenų parinkimo metodas leidžia generuoti skirtingomis savybėmis pasižyminčius duomenų rinkinius, jis yra naudingas išbandant mašininio mokymosi modelius su skirtingomis savybėmis pasižyminičiais duomenim.     
        
        Duomenų generavimui yra naudojamos etalono (angl.~\textit{benchmark}) funkcijos. Jų yra daugybė, todėl žemiau yra paminėtos tik kelios dažniausiai naudojamos funkcijos \cite{naser_benchmark_funcs}.
        
        Sferos funkcija:

        \begin{equation}
            \label{eq:sferos_etalono_funkc}
            f(X) = \sum_{i = 0}^{n}{x_i^2}
        \end{equation}

        Rastrigin funkcija:

        \begin{equation}
            \label{eq:rastrigin_etalono_funkc}
            f(X) = An + \sum_{i = 1}^{n}{(x_i - A\cos{2\pi x_i})}
        \end{equation}

        Griewank funkcija:

        \begin{equation}
            \label{eq:griewank_etalono_funkc}
            \begin{aligned}
                f(X) &= 1 + \frac{1}{4000} \sum_{i = 1}^{n} x_{i}^{2} - \prod_{i = 1}^{n} P_i(x_i), \\
                P_i(x_i) &= \cos\!\left(\frac{x_i}{\sqrt{i}}\right)
            \end{aligned}
        \end{equation}


    \subsection{Triukšmo duomenų aibei pritaikymas}

        Dirbant su realaus pasaulio duomenimis dažnai susiduriama su tuo, kad jie nėra visiškai tikslūs. To priežastys gali būti įvairios -- žmogaus klaidos, prietaisų tikslumo ribos, aplinkos veiksnių poveikis ir t.t. Tokie nedideli netikslumai duomenyse yra vadinimai triukšmu \cite{medium_noise_types}.

        Dirbtinių neuroninių tinklų modelių apmokymo kontekste, triukšmo pridėjimas yra plačiai taikomas metodas, naudojamas dėl įvairių priežasčių. Viena pagrindinių priežasčių yra siekis sumažinti modelio persimokymo (angl.~\textit{overfitting}) riziką, kai modelis pernelyg tiksliai prisitaiko prie apmokymo duomenų ir praranda gebėjimą apibendrinti naujus, anksčiau nematytus duomenis. Triukšmas taip pat gali būti naudojamas modelio atsparumui įvertinti ir padidinti, yptaingai tais atvejais, kai realiose taikymo sąlygose įvesties duomenys yra paveikti įvairių trikdžių ar neapibrėžtumo \cite{gfg_noise_injection}.

        \paragraph{Gauso triukšmas}

            Gauso triukšmas (angl.~\textit{Gaussian Noise}) -- tai triukšmas generuojamas pagal plačiai žinomą Gauso (normalųjį) skirstinį (\ref{img:gauso_triuksmas} pav.). Jis yra apibrėžiamas dviem pagrindiniais parametrais: vidurkiu ($\mu$) ir standartiniu nuokrypiu ($\sigma$). Šiame kontekste parametras $\mu$ nusako vidutinę triukšmo paklaidą, o standartinis nuokrypis $\sigma$ –- vidutinį triukšmo reikšmių išsisklaidymą į abi puses nuo vidurkio.

            \begin{figure}[H]
                \centering
                \includegraphics[width=0.7\linewidth]{attachments/img/gauso_triuksmas.png}
                \caption{Gauso triukšmas, pritaikytas tiesinės funkcijos reikšmėms ($\mu = 0$, $\sigma = 0,5$).}
                \label{img:gauso_triuksmas}
            \end{figure}

        \paragraph{Druskos ir pipirų triukšmas}

            Druskos ir pipirų triukšmas (angl.~\textit{Salt and Pepper Noise}) -- tai triukšmas, pridėtas atsitiktinai parenkant duomenų aibės taškus ir jų reikšmes pakeičiant į minimalią arba maksimalią duomenų aibės reikšmę (\ref{img:druskos_ir_pipiru_triuksmas} pav.). Pritaikius šį triukšmo pridėjimo metodą, duomenyse atsiranda staigūs šuoliai. Pagrindinis šio metodo parametras yra taškų dalis, kuriai pritaikomas triukšmas ($p$).

            \begin{figure}[H]
                \centering
                \includegraphics[width=0.7\linewidth]{attachments/img/druskos_ir_pipiru_triuksmas.png}
                \caption{Druskos ir pipirų triukšmas, pritaikytas tiesinės funkcijos reikšmėms ($p = 10\%$).}
                \label{img:druskos_ir_pipiru_triuksmas}
            \end{figure}

        \paragraph{Kvantavimo triukšmas}

            Kvantavimo triukšmas (angl.~\textit{Quantization Noise}) -- tai triukšmas, pridedamas atsitiktinai pasirenkant aibės taškus ir supavalininant jų reikšmes (\ref{img:islyginimo_triuksmas} pav.). Šio metodo rezultatas irgi lemia nedidelius duomenų šuolius, tačiau jie yra arčiau originalių reikšmių. Šio metodo parametras yra duomenų aibės dalis, kurioms yra pritaikomas triukšmas ($p$).

            \begin{figure}[H]
                \centering
                \includegraphics[width=0.7\linewidth]{attachments/img/islyginimo_triuksmas.png}
                \caption{Kvantavimo triukšmas, pritaikytas tiesinės funkcijos reikšmėms ($p = 10\%$).}
                \label{img:islyginimo_triuksmas}
            \end{figure}