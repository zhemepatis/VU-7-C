\subsubsection{Gauso skirstinio metodas}

    Vienas iš triukšmo pridėjimo būdų pasitelkia Gauso skirstinį.

    Žemiau pateiktuose paveikslėliuose galime pamatyti, kaip $\sigma$ triukšmo parametro dydis lemia taškų išsidėstymą.

    Žemiau pateiktame paveikslėlyje matome triukšmo pritaikymo pavyzdį dvimatėje erdvėje. 

    \begin{figure}[H]
        \centering
        \includegraphics[width=0.8\linewidth]{attachments/img/triuksmo_pvz_0.png}
        \caption{Švarių ir duomenų su triukšmu sugretinimas, kai ...}
        \label{pav:triuksmo_ir_svariu_duomenu_sugretinimas_0}
    \end{figure}

    \begin{figure}[H]
        \centering
        \includegraphics[width=0.8\linewidth]{attachments/img/triuksmo_pvz_1.png}
        \caption{Švarių ir duomenų su triukšmu sugretinimas, kai ...}
        \label{pav:triuksmo_ir_svariu_duomenu_sugretinimas_1}
    \end{figure}

    Šio kursinio kontekste paklaidos buvo generuojamas pasitelkiant normalųjį skirstinį, kur paklaidos vidurkis yra 0, o standartinis nuokrypis 0.5.