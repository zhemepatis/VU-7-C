\section{Dirbtinių neuroninių tinklų pritaikymas}
\label{skr:dnt_pritaikymas}

    Dirbtiniai neuroniniai tinklai (angl. \enquote{Artificial Neural Networks}) - mašininio mokymosi sritis, apimanti skaitinius modelius, pasižyminčius sluoksnine architektūra (angl. \enquote{layered architecture}) bei savarankišku mokymusi, siekiant nustatyti matematinį sąryšį tarp duomenų įvesties ir numatytos išvesties. Kadangi dirbtiniai neuroniniai tinklai taip pat pasižymi netiesiškumu ir sugebėjimu apibendrinti duomenis, šie modeliai yra tinkami naudoti užduotyse, reikalaujančiose sudėtingų raštų (angl. \enquote{pattern}) atpažinimo. Šios užduotys dažniausiai yra skirstomos į dvi rūšis: klasifikavimą ir regresiją. Daugiau apie kiekvieną iš jų galima rasti atitinkamai \ref{skr:klasifikavimas} ir \ref{skr:regresija} poskyriuose.

    Šio kursinio projektinio darbo metu buvo pasirinkta spręsti regresijos problemą, siekiant įvertinti dirbtinių neuroninių tinklų apmokymo sąlygas, kai mokymosi duomenys nėra visiškai tikslūs ir turi triukšmo.

    \subsection{Klasifikavimas}
\label{skr:klasifikavimas}

Klasifikavimas - užduotis, kurioje modelis iš numatytos diskrečios aibės turi kiekvienam duomenų įrašui priskirti vieną ar kelias reikšmes. Pavyzdžiui, klasifikavimo uždavinio pavyzdžiu galėtų būti klasių \enquote{katė} arba \enquote{šuo} priskyrimas pateiktoms nuotraukoms. Kitas pavyzdys galėtų būti augalo rūšies nustatymas pagal duotus konkretaus objekto parametrus: žiedo spalvą, lapų matmenis ir pan. 
    \subsection{Regresija}
\label{skr:regresija}

    Regresijos problemą sprendžiantys modeliai yra susitelkę ties tolydžios reikšmės nustatymu. Šio tipo uždavinių pavyzdžiai yra finansų rinkos statistinių rodiklių nuspėjimas, temperatūros prognozė, remiantis istoriniais duomenimis ir t.t.