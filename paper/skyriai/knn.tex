\section{K artimiausių kaimynų metodas}

    K artimiausių kaimynų (angl.~\textit{K Nearest Neighbors}) metodas yra vienas paprasčiausių prižiūrimo mokymosi (angl.~\textit{supervised learning}) algoritmų. Jis yra grindžiamas sprendimų priėmimu, remiantis artimiausiai esančių taškų informacija \cite{cover_knn}.

    \subsection{Eiga}

        K artimiausių kaimynų metodas priskiriamas vadinamiesiems tingiesiems algoritmams, nes jis nėra apmokomas iš anksto – visi skaičiavimai atliekami tik prognozavimo metu.

        Klasifikavimo atveju, gavus naują objektą, apskaičiuojami atstumai tarp šio objekto ir visų mokymo duomenų įrašų. Tuomet parenkami $k$ artimiausių kaimynų, o objektui priskiriama klasė nustatoma pagal daugumos balsavimo principą, t. y. pasirenkama ta klasė, kuriai priklauso daugiausia kaimynų.

        Regresijos uždaviniuose metodas veikia panašiai, tačiau galutinė prognozė apskaičiuojama kaip $k$ artimiausių kaimynų reikšmių vidurkis. Tokiu būdu gaunama tolydi prognozuojama reikšmė, atspindinti artimiausių duomenų taškų savybes.

    \subsection{Parametrai}

        Iš algoritmo veikimo principo matyti, kad pagrindiniai jo parametrai yra kaimynų skaičius $k$ ir atstumo skaičiavimo metodas. Kaimynų skaičius $k$ apibrėžia, kiek artimiausių taškų bus įtraukiama į sprendimo priėmimą, o atstumo skaičiavimo metodas nusako, kaip bus vertinamas atstumas tarp duomenų taškų. Toliau pateikiamos dažniausiai taikomos atstumo skaičiavimo formulės (\ref{eq:euklido_atstumas}, \ref{eq:manhatano_atstumas} ir \ref{eq:minkovskio_atstumas} formulės).
                
        Euklido (angl.~\textit{Euclidean}) atstumo formulė:
        \begin{equation}
            \label{eq:euklido_atstumas}
            D(A, B) = \sqrt{\sum_{i = 1}^{n}{(A_i - B_i)^2}}
        \end{equation}

        Manhatano (angl.~\textit{Manhattan}) atstumo formulė:
        \begin{equation}
            \label{eq:manhatano_atstumas}
            D(A, B) = \sum_{i = 1}^{n}{\mid A_i - B_i \mid}
        \end{equation}

        Minkovskio (angl.~\textit{Minkowski}) atstumo formulė:

        \begin{equation}
            \label{eq:minkovskio_atstumas}
            D(A, B)_p = \big( \sum_{i = 1}^{n} {\mid A_i - B_i \mid^p} \big) ^ {\frac{1}{p}}
        \end{equation}