\sectionnonum{Išvados}
    
    Šiame kursiniame darbe buvo apžvelgtos pagrindinės sąvokos, susijusios su dirbtinių neuroninių tinklų kūrimu ir apmokymu. Taip pat buvo aptartas $k$ artimiausių kaimynų metodas, kuris darbe buvo taikomas dirbtinio neuroninio tinklo rezultatų įvertinimui. Be to, reikšmingas dėmesys buvo skirtas duomenų aibių sudarymui ir jų paruošimui eksperimentiniams tyrimams.
    
    Atlikus bandymus su dirbtiniu neuroniniu tinklu ir $k$ artimiausių kaimynų metodu, nustatyti aiškūs šių modelių elgsenos skirtumai, ypač vertinant triukšmo įtaką modelių tikslumui ir stabilumui. Analizuojant tiek švarias, tiek triukšmingas duomenų aibes, buvo padarytos tos pačios išvados:

    \begin{itemize}
        \item dirbtinis neuroninis tinklas visais nagrinėtais atvejais pasižymėjo mažesnėmis absoliutinės paklaidos vidurkio ir standartinio nuokrypio reikšmėmis;
        \item $k$ artimiausių kaimynų metodo atveju nustatyta, kad didėjant mokymo duomenų aibės apimčiai, modelio tikslumas didėjo nuosekliai ir panašiu tempu;
        \item dirbtinio neuroninio tinklo atveju tikslumo rodiklių pokyčiai buvo mažiau stabilūs, o kai kuriais atvejais stebėtas ir jų pablogėjimas.
    \end{itemize}

    Vertinant triukšmo įtaką modelių rezultatams, buvo pastebėtos šios tendencijos:

    \begin{itemize}
        \item $k$ artimiausių kaimynų metodas demonstravo nuoseklų triukšmo poveikio didėjimą, didėjant duomenų aibės apimčiai;
        \item dirbtinio neuroninio tinklo atveju kai kuriais atvejais buvo stebėtas reikšmingai didesnis triukšmo poveikis, lyginant su $k$ artimiausių kaimynų metodo rezultatais.
    \end{itemize}

    Ateities darbuose būtų tikslinga atlikti išsamesnę triukšmo poveikio analizę, keičiant ne tik triukšmo intensyvumo parametrus, bet ir patį triukšmo tipą. Taip pat būtų prasminga išplėsti tyrimą, nagrinėjant triukšmo įtaką skirtingo pobūdžio duomenų aibėms, siekiant įvertinti gautų rezultatų bendrumą. Be to, būsimuose tyrimuose būtų naudinga analizuoti triukšmo poveikį įvairioms dirbtinių neuroninių tinklų architektūroms, siekiant nustatyti architektūros pasirinkimo reikšmę modelio triukšmo atsparumui. Analogiškai, atsižvelgiant į tai, kad šiame darbe buvo nagrinėtas $k$ artimiausių kaimynų metodas su fiksuotomis $k = 1$ ir $k = 3$ reikšmėmis, tolimesniuose darbuose būtų tikslinga ištirti, kaip triukšmo poveikis kinta priklausomai nuo parametro $k$ pasirinkimo.