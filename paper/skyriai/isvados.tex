\sectionnonum{Išvados}
    
    Šiame kursiniame darbe buvo apžvelgtos pagrindinės sąvokos, susijusios su dirbtinių neuroninių tinklų kūrimu ir apmokymu. Taip pat buvo aptartas k artimiausių kaimynų metodas, kuris darbe buvo taikomas dirbtinio neuroninio tinklo rezultatų įvertinimui. Be to, reikšmingas dėmesys buvo skirtas duomenų aibių sudarymui ir jų paruošimui eksperimentiniams tyrimams.
    
    Be literatūros apžvalgos, darbe buvo atlikti praktiniai tyrimai, kurių metu buvo sukurti ir pritaikyti dirbtinio neuroninio tinklo bei k artimiausių kaimynų metodo modeliai. Tyrimų metu buvo analizuojama šių modelių tikslumo priklausomybė nuo mokymo duomenų aibės dydžio ir triukšmo lygio, naudojant tiek švarias, tiek triukšmingas duomenų aibes. Gauti rezultatai leido įvertinti modelių atsparumą triukšmui ir palyginti jų elgseną skirtingomis eksperimentinėmis sąlygomis.

    Ateities darbuose būtų tikslinga atlikti platesnę triukšmo įtakos analizę, keičiant tiek triukšmo intensyvumo parametrus, tiek patį triukšmo tipą. Taip pat būtų prasminga ištirti triukšmo poveikį kitokio pobūdžio duomenų aibėms, siekiant įvertinti gautų rezultatų bendrumą. Be to, ateities tyrimuose rekgalima būtų nagrinėti triukšmo įtaką skirtingoms dirbtinių neuroninių tinklų architektūroms, siekiant nustatyti architektūros pasirinkimo reikšmę modelio atsparumui triukšmui.