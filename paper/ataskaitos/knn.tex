\subsubsection{Sferos funkcija, kai n = 3}


\subsubsubsection{Palyginimas}



% -----

% Žemiau esančiose lentelėse \ref{lnt:nn_sferos_funkc_paklaidos_rodikliai} ir \ref{lnt:knn_sferos_funkc_paklaidos_rodikliai} galima pamatyti modelių spėjimų absoliutinių paklaidų statistinius rodiklius. Prieš skaičiuojant šiuos rodiklius, modelių spėjimai buvo konvertuoti į originalų duomenų mastelį. Taigi atsižvelgus, kad tyrimo metu duomenys buvo apibrėžiami sferos funkcija, maksimali galima absoliutinė paklaida šio tyrimo atveju yra 100.