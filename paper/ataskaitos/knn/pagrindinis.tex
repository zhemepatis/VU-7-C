\subsection{K artimiausių kaimynų metodo rezultatai}

Tyrimas buvo atliktas su 1~000, 10~000, 100~000, 1~000~000 ir 10~000~000 taškų aibėmis, kurių $85\%$ buvo skiriami modelio apmokymui ir validavimui, likę $15\%$ - modelio testavimui.

Bandymui buvo generuojami keturmačiai duomenys ($n = 4$), kurių kiekviena komponentė buvo atsitiktinai parenkama iš intervalo $[-5; 5]$. Prieš apmokant modelį, duomenys buvo papildomai apdorojami: kiekviena vektoriaus komponentė bei sferos funkcijos reikšmės buvo normalizuojamos naudojant min–max metodą.

Modelis buvo validuojamas pasitelkiant kryžminę validaciją (angl. \enquote{cross-validation}). Bandymų metu buvo nustatyta, kad geriausiai modelis veikia, kai kaimynų skaičius yra lygus $3$, tačiau palyginimui buvo atlikti bandymai ir kai artimiausias kaimynas yra vienas.

Visais atvejais prieš skaičiuojant statistinius absoliutinės paklaidos rodiklius, modelio spėjimai buvo konvertuoti į originalų duomenų mastelį. Taigi atsižvelgus, kad tyrimo metu duomenys buvo apibrėžiami sferos funkcija, maksimali galima absoliutinė paklaida šio tyrimo atveju yra 100.

\subsubsection{Sferos funkcija, kai k = 1}

\subsubsubsection{Švarūs duomenys}

    Šiame bandyme buvo tiriamas K artimiausių kaimynų metodo gebėjimas interpoliuoti duomenis, kurie pasižymi sferos funkcijos savybėmis. Šiuo atveju $k = 1$.

    \begin{table}[H]
    \centering
    
    \begin{tabular}{|c|c|c|c|c|}
        \hline
        \textbf{Taškų kiekis} & \textbf{Minimumas} & \textbf{Maksimumas} & \textbf{Vidurkis} & \textbf{Standartinis nuokrypis} \\ \hline
        1 000      & 0.050320 & 20.178803 & 5.182272 & 3.617230 \\ \hline
        10 000     & 0.002558 & 13.502786 & 2.975179 & 2.329911 \\ \hline
        100 000    & 0.000256 & 11.192788 & 1.698232 & 1.322762 \\ \hline
        1 000 000  & 0.000007 & 6.239792 & 0.959560 & 0.743995 \\ \hline
        10 000 000  & 0.00000049 & 3.839611 & 0.538008 & 0.413742 \\ \hline
    \end{tabular}
    
    \caption{K artimiausių kaimynų metodo absoliutinės paklaidos statistiniai rodikliai nagrinėjant sferos funkciją (k = 1)}
    \label{lnt:nn_sferos_funkc_rodikliai_0}
\end{table}

\subsubsubsection{Triukšmingi duomenys}

Šiame bandyme buvo tiriamas K artimiausių kaimynų metodo gebėjimas interpoliuoti duomenis, kurie pasižymi sferos funkcijos savybėmis, jiems pridėjus triukšmo. Šiuo atveju $k = 1$.

\begin{table}[H]
    \centering
    
    \begin{tabular}{|c|c|c|c|c|}
        \hline
        \textbf{Taškų kiekis} & \textbf{Minimumas} & \textbf{Maksimumas} & \textbf{Vidurkis} & \textbf{Standartinis nuokrypis} \\ \hline
        1 000        & 0.028075 & 20.031243 & 5.220319 & 3.692268 \\ \hline
        10 000       & 0.007079 & 14.064025 & 2.997375 & 2.348190 \\ \hline
        100 000      & 0.000221 & 10.859588 & 1.743053 & 1.354251 \\ \hline
        1 000 000    & 0.000006 & 6.902217  & 1.042092 & 0.800102 \\ \hline
        10 000 000   & 0.000000 & 4.408415  & 0.670634 & 0.510044 \\ \hline
    \end{tabular}
    
    \caption{K artimiausių kaimynų metodo absoliutinės paklaidos statistiniai rodikliai nagrinėjant sferos funkciją (k = 1)}
    \label{lnt:nn_sferos_funkc_rodikliai_1}
\end{table}

\subsubsubsection{Palyginimas}

Iš pateiktų \ref{lnt:nn_sferos_funkc_rodikliai_0} ir \ref{lnt:nn_sferos_funkc_rodikliai_1} lentelių matome, kad pridėjus triukšmą k artimiausių kaimynų interpeliacijos rezultatų tikslumas žymiai nesikeitė. Didžiausias absoliutinės paklaidos vidurkio padidėjimas siekia 1,2 karto. Tokias pačias įžvalgas galima padaryti ir nagrinėjant standartinį nuokrypį. 

\subsubsection{Sferos funkcija, kai k = 3}

\input{ataskaitos/knn/knn_rodikliai_0}
\subsubsubsection{Triukšmingi duomenys}

    Šiame bandyme buvo tiriamas K artimiausių kaimynų metodo gebėjimas interpoliuoti duomenis, kurie pasižymi sferos funkcijos savybėmis. Šiuo atveju $k = 3$.

    \input{attachments/tables/knn_sferos_funkc_rodikliai_1}
\subsubsubsection{Palyginimas}

    Iš aukščiau esančių \ref{tbl:knn_sferos_funkc_svarus} ir \ref{tbl:knn_sferos_funkc_su_triuksmu} lentelių galima pamatyti, kad taikant artimiausių kaimynų metodą, kai $k = 3$ tikslumas mažėja neženkliai - didžiausias absoliutinės paklaidos vidurkio padidėjimas siekia 1,2 karto. Standartinio nuokrypis padidėja iki 1,1 karto.

\subsubsection{K artimiausių kaimynų metodo rezultatų palyginimas}

Palyginus metodo rezultatus, kai $k = 1$ ir $k = 3$ yra akivaizdu, kad su didesniu kaimynų kiekiu, modelis grąžina geresnius rezultatus. Taip pat matome, kad abiem atvejais, modelių grąžinami rezultatai stabiliai gerėjo.
