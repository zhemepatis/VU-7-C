\section{Išvados ir pasiūlymai tolimesniem darbam}

    \begin{frame}[allowframebreaks]{Išvados ir pasiūlymai tolimesniems darbams}

        \begin{itemize}
            
            \item Analizuojant tiek švarias, tiek triukšmingas duomenų aibes, buvo padarytos tos pačios išvados:

            \begin{itemize}
                \item dirbtinis neuroninis tinklas visais nagrinėtais atvejais pasižymėjo mažesnėmis absoliutinės paklaidos vidurkio ir standartinio nuokrypio reikšmėmis;
                \item $k$ artimiausių kaimynų metodo atveju nustatyta, kad didėjant mokymo duomenų aibės apimčiai, modelio tikslumas didėjo nuosekliai ir panašiu tempu;
                \item dirbtinio neuroninio tinklo atveju tikslumo rodiklių pokyčiai buvo mažiau stabilūs, o kai kuriais atvejais stebėtas ir jų pablogėjimas.
            \end{itemize}

            \break

            \item Lyginant, kaip įterptas triukšmas paveikė modelių tikslumo rodiklius, buvo nustatytos šios tendencijos:

            \begin{itemize}
                \item didėjant duomenų aibės apimčiai, $k$ artimiausių kaimynų metodui būdingas vis didesnis tikslumo rodiklių skirtumas tarp švarių ir triukšmingų duomenų aibių;
                \item dirbtinio neuroninio tinklo atveju buvo stebėtas reikšmingai didesnis tikslumo rodiklių pablogėjimas, lyginant su $k$ artimiausių kaimynų metodo rezultatais.
            \end{itemize}

            \break

            \item Siūlymai ateities darbam:
            
            \begin{itemize}
                \item atlikti išsamesnę triukšmo poveikio analizę, keičiant triukšmo intensyvumo parametrus ir triukšmo tipą;
                \item pritaikyti modelius kitokio pobūdžio aibėms, siekiant įvertinti gautų rezultatų bendrumą;
                \item išnagrinėti triukšmo poveikį įvarioms dirbtinių neuroninių tinklų architektūroms;
                \item ištirti, kaip triukšmo poveikis $k$ artimiausių kaimynų modeliui kinta priklausomai nuo parametro $k$ pasirinkimo.
            \end{itemize}
            
        \end{itemize}
        
    \end{frame}