\begin{frame}[allowframebreaks]{Išvados ir pasiūlymai tolimesniems darbams}

    \begin{itemize}
        
        \item Analizuojant tiek švarias, tiek triukšmingas duomenų aibes, buvo padarytos tos pačios išvados:

        \begin{itemize}
            \item dirbtinis neuroninis tinklas visais nagrinėtais atvejais pasižymėjo mažesnėmis absoliutinės paklaidos vidurkio ir standartinio nuokrypio reikšmėmis;
            \item $k$ artimiausių kaimynų metodo atveju nustatyta, kad didėjant mokymo duomenų aibės apimčiai, modelio tikslumas didėjo nuosekliai ir panašiu tempu;
            \item dirbtinio neuroninio tinklo atveju tikslumo rodiklių pokyčiai buvo mažiau stabilūs, o kai kuriais atvejais stebėtas ir jų pablogėjimas.
        \end{itemize}

        \item Lyginant, kaip įterptas triukšmas paveikė modelių tikslumo rodiklius, buvo nustatytos šios tendencijos:

        \begin{itemize}
            \item didėjant duomenų aibės apimčiai, $k$ artimiausių kaimynų metodui būdingas vis didesnis tikslumo rodiklių skirtumas tarp švarių ir triukšmingų duomenų aibių;
            \item dirbtinio neuroninio tinklo atveju buvo stebėtas reikšmingai didesnis tikslumo rodiklių pablogėjimas, lyginant su $k$ artimiausių kaimynų metodo rezultatais.
        \end{itemize}

        \item Ateities darbuose būtų tikslinga atlikti išsamesnę triukšmo poveikio analizę, keičiant ne tik triukšmo intensyvumo parametrus, bet ir patį triukšmo tipą. Taip pat būtų prasminga išplėsti tyrimą, nagrinėjant triukšmo įtaką skirtingo pobūdžio duomenų aibėms, siekiant įvertinti gautų rezultatų bendrumą. Be to, būsimuose tyrimuose būtų naudinga analizuoti triukšmo poveikį įvairioms dirbtinių neuroninių tinklų architektūroms, siekiant nustatyti architektūros pasirinkimo reikšmę modelio triukšmo atsparumui. Analogiškai, atsižvelgiant į tai, kad šiame darbe buvo nagrinėtas $k$ artimiausių kaimynų metodas su fiksuotomis $k = 1$ ir $k = 3$ reikšmėmis, tolimesniuose darbuose būtų tikslinga ištirti, kaip triukšmo poveikis kinta priklausomai nuo parametro $k$ pasirinkimo.

    \end{itemize}
    
\end{frame}