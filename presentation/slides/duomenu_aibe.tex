\section{Duomenų aibė}

    \begin{frame}[allowframebreaks]{Duomenų aibė}

        \begin{itemize}
            
            \item Bandymams atlikti buvo generuojami keturmačiai duomenys, kurių kiekviena komponentė buvo parenkama iš intervalo $[-5; 5]$. Duomenys buvo generuoti naudojant sferos etalono funkciją (\ref{eq:sferos_etalono_funkc} formulė) \cite{naser_benchmark_funcs}.

            \begin{equation}
                \label{eq:sferos_etalono_funkc}
                f(X) = \sum_{i = 0}^{n}{x_i^2}
            \end{equation}

            \item Triukšmo pridėjimui į duomenų aibę buvo pasitelktas Gauso paskirsymo funkcijos metodas. Triukšmas buvo generuojamos su parametrais $\mu = 0$, $\sigma = 0.5$. Paklaidos buvo pridedamos prie sugeneruotų etalono funkcijų reikšmių.

            \break
            
            \item Dirbtinis neuroninis tinklas buvo pritaikytas su 1~000, 10~000, 100~000, 1~000~000 dydžio duomenų aibėmis, tuo tarpu $k$ artimiausių kaimynų metodas dar papildomai buvo išbandytas su aibe, turinčia 10~000~000 įrašų.
            
            \item Prieš apmokant modelius duomenys buvo normalizuojami min-max metodu. Perėjus prie modelių statistinių rodiklių įvertinimo duomenys buvo grąžinami į pradinį duomenų mąstelį. Tad atsižvelgus į tai, kad duomenys buvo generuoti sferos funkcija prognozės absoliutinė paklaida įeina į intervalą $[0; 100]$.

        \end{itemize}
        
    \end{frame}