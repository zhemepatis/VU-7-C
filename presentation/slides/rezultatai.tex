\section{Tyrimo rezultatai}

    \begin{frame}[allowframebreaks]{Dirbtinio neuroninio tinklo rezultatai}

        \begin{table}[H]
    \centering
    
    \begin{tabular}{|c|c|c|c|c|}
        \hline
        \textbf{Taškų kiekis} & \textbf{Minimumas} & \textbf{Maksimumas} & \textbf{Vidurkis} & \textbf{Standartinis nuokrypis} \\ \hline
        1 000      & 0.002857 & 6.881401 & 1.773480 & 1.231157 \\ \hline
        10 000     & 0.001137 & 4.690506 & 0.625588 & 0.514914 \\ \hline
        100 000    & 0.000001 & 1.170944 & 0.217896 & 0.172228 \\ \hline
        1 000 000  & 0.000004 & 1.631981 & 0.416016 & 0.274041 \\ \hline
    \end{tabular}
    
    \caption{Dirbtinio neuroninio tinklo rezultatai su švariais sferos funkcijos generuotais duomenis}
    \label{tbl:fnn_sferos_funkc_svarus}
\end{table}


        \break

        \begin{itemize}

            \item Didžiąja dalimi, didinant taškų kiekį, modelio absoliutinės paklaidos vidurkis ir standartinis nuokrypis mažėjo. Išsiskiria atvejis su 1~000~000 taškų aibe - modelis grąžino blogesnius rezultatus lyginant su rezultatais, gautais naudojant mažėsnę taškų aibę. 
            
            \item Standartinio nuokrypio reikšmės pasižymi analogišku dėsningumu.

            \item Abiem rodiklių gerėjimo atvejais absoliučios paklaidos vidurkis sumažėjo atitinkamai 64,7\% ir 65,1\%. Standartinis nuokrypis, didinant duomenų aibės dydį nuo 1~000 iki 10~000, sumažėjo 58,1\%, o nuo 1~000 iki 100~000 -- 66,6\%.        

        \end{itemize}

        \break

        \begin{table}[H]
    \centering
    
    \begin{tabular}{|c|c|c|c|c|}
        \hline
        \textbf{Taškų kiekis} & \textbf{Minimumas} & \textbf{Maksimumas} & \textbf{Vidurkis} & \textbf{Standartinis nuokrypis} \\ \hline
        1 000      & 0.008972 & 9.575722 & 1.970832 & 1.910026 \\ \hline
        10 000     & 0.000362 & 5.672182 & 2.127527 & 1.227688 \\ \hline
        100 000    & 0.000038 & 2.539330 & 0.536711 & 0.420499 \\ \hline
        1 000 000  & 0.000006 & 1.809952 & 0.419722 & 0.278947 \\ \hline
    \end{tabular}
    
    \caption{Dirbtinio neuroninio tinklo rezultatai su triukšmingais duomenim}
    \label{tbl:fnn_sferos_funkc_su_triuksmu}
\end{table}


        \break

        \begin{itemize}

            \item Didinant aibės dydį, daugeliu atveju, absoliutinės paklaidos vidurkis mažėjo. Vienintėlis atvejis, kai didinant duomenų aibę vidurkio reikšmė suprastėjo, buvo keičiant įrašų kiekį iš 1~000 į 10~000.
            
            \item Standartinio nuokrypio reikšmes, matome, kad šis statistinis rodiklis šiuo atveju stabiliai mažėjo

            \item Didinant taškų skaičių nuo 10~000 iki 100~000, vidurkis sumažėjo 74,8\%, o keičiant aibę iš 100~000 į 1~000~000, vidurkis sumažėjo apie 21,8\%. Tuo tarpu standartinis nuokrypis, keičiant įrašų skaičių iš 1~000 į 10~000, iš 10~000 į 100~000 bei iš 100~000 į 1~000~000, sumažėjo atitinkamai 35,7\%, 65,8\% ir 33,7\%.

        \end{itemize}

        \break

        \begin{itemize}
            
            \item Mažiausią įtaką triukšmas turėjo duomenų aibei su 1~000~000 taškų -- modelio absoliučios paklaidos vidurkis padidėjo 0,9\%, o standartinis nuokrypis -- 1,8\%. 
            
            \item Bandymuose su 1~000 įrašų duomenų aibe, vidurkis padidėjo 11,1\%, o standartinis nuokrypis -- 55,1\%. 
            
            \item Didžiausi pakitimai pastebimi nagrinėjant 10~000 ir 100~000 įrašų duomenų aibes: šiose aibėse vidurkis padidėjo atitinkamai 240\% ir 146,3\%, o standartinis nuokrypis -- 138,4\% ir 144,1\%.

        \end{itemize}

    \end{frame}

    \begin{frame}[allowframebreaks]{$k$ artimiausių kaimynų metodo rezultatai ($k = 1$)}

        \input{attachments/tables/nn_sferos_funkc_svarus}
        \begin{table}[H]
    \centering
    
    \begin{tabular}{|c|c|c|c|c|}
        \hline
        \textbf{Taškų kiekis} & \textbf{Minimumas} & \textbf{Maksimumas} & \textbf{Vidurkis} & \textbf{Standartinis nuokrypis} \\ \hline
        1 000        & 0,235338 & 24,587910 & 7,197508 & 5,597238 \\ \hline
        10 000       & 0,009761 & 27,517120 & 5,291629 & 4,354544 \\ \hline
        100 000      & 0,000215 & 41,895482 & 4,772135 & 3,972952 \\ \hline
        1 000 000    & 0,000034 & 35,674550 & 4,636560 & 3,832948 \\ \hline
        10 000 000   & 0,000010 & 40,702695 & 4,576937 & 3,782748 \\ \hline
    \end{tabular}
    
    \caption{K artimiausių kaimynų metodo rezultatai su triukšmingais duomenim (k = 1)}
    \label{tbl:nn_sferos_funkc_su_triuksmu}
\end{table}


    \end{frame}

    \begin{frame}[allowframebreaks]{$k$ artimiausių kaimynų metodo rezultatai ($k = 3$)}

        \begin{table}[H]
    \centering
    
    \begin{tabular}{|c|c|c|c|c|}
        \hline
        \textbf{Taškų kiekis} & \textbf{Minimumas} & \textbf{Maksimumas} & \textbf{Vidurkis} & \textbf{Standartinis nuokrypis} \\ \hline
        1 000      & 0.051417 & 22.357332 & 4.055465 & 3.877167 \\ \hline
        10 000     & 0.001065 & 12.780683 & 2.314161 & 1.924742 \\ \hline
        100 000    & 0.000089 & 10.150919 & 1.240283 & 1.036862 \\ \hline
        1 000 000  & 0.000011 & 5.804630  & 0.677661 & 0.562266 \\ \hline
        10 000 000  & 0.0000001  & 3.9141042 & 0.373981 & 0.306836 \\ \hline
    \end{tabular}
    
    \caption{K artimiausių kaimynų metodo absoliutinės paklaidos statistiniai rodikliai nagrinėjant sferos funkciją (k = 3)}
    \label{tbl:knn_sferos_funkc_svarus}
\end{table}

        \begin{table}[H]
    \centering
    
    \begin{tabular}{|c|c|c|c|c|}
        \hline
        \textbf{Taškų kiekis} & \textbf{Minimumas} & \textbf{Maksimumas} & \textbf{Vidurkis} & \textbf{Standartinis nuokrypis} \\ \hline
        1 000        & 0,030330   & 22,659118  & 4,040047   & 3,888274   \\ \hline
        10 000       & 0,001735   & 12,557448  & 2,314843   & 1,926788   \\ \hline
        100 000      & 0,000121   & 10,204885  & 1,263033   & 1,047999   \\ \hline
        1 000 000    & 0,000002   & 5,948286   & 0,718320   & 0,585065   \\ \hline
        10 000 000   & $<10^{-6}$ & 4,096856   & 0,442222   & 0,348677   \\ \hline
    \end{tabular}
    
    \caption{K artimiausių kaimynų metodo rezultatai su triukšmingais duomenim (k = 3).}
    \label{tbl:knn_sferos_funkc_su_triuksmu}
\end{table}


    \end{frame}