\begin{frame}[allowframebreaks]{Tyrimo rezultatai}

    \begin{itemize}

        \item Nagrinėjant modelio rezultatus su švariais duomenim (\ref{tbl:fnn_sferos_funkc_svarus} lentelė) matome, kad, didžiąja dalimi, didinant taškų kiekį, modelio absoliutinės paklaidos vidurkis ir standartinis nuokrypis mažėjo. Išsiskiria tik atvejis su 1~000~000 taškų aibe - modelis grąžino blogesnius rezultatus lyginant su rezultatais, gautais naudojant 100~000 taškų aibę. Žvelgiant į standartinio nuokrypio reikšmes, matome analogišką dėsningumą.

        \item Lyginant modelio rodiklius skirtingam taškų kiekiui, matyti, kad abiem gerėjimo atvejais absoliučios paklaidos vidurkis sumažėjo atitinkamai 64,7\% ir 65,1\%. Tuo tarpu standartinis nuokrypis, didinant duomenų aibės dydį nuo 1~000 iki 10~000, sumažėjo 58,1\%, o nuo 1~000 iki 100~000 -- 66,6\%.        
        
        \begin{table}[H]
    \centering
    
    \begin{tabular}{|c|c|c|c|c|}
        \hline
        \textbf{Taškų kiekis} & \textbf{Minimumas} & \textbf{Maksimumas} & \textbf{Vidurkis} & \textbf{Standartinis nuokrypis} \\ \hline
        1 000      & 0.002857 & 6.881401 & 1.773480 & 1.231157 \\ \hline
        10 000     & 0.001137 & 4.690506 & 0.625588 & 0.514914 \\ \hline
        100 000    & 0.000001 & 1.170944 & 0.217896 & 0.172228 \\ \hline
        1 000 000  & 0.000004 & 1.631981 & 0.416016 & 0.274041 \\ \hline
    \end{tabular}
    
    \caption{Dirbtinio neuroninio tinklo rezultatai su švariais sferos funkcijos generuotais duomenis}
    \label{tbl:fnn_sferos_funkc_svarus}
\end{table}


        \item Pažvelgus į modelio rezultatus su triukšmingais duomenim (\ref{tbl:fnn_sferos_funkc_su_triuksmu} lentelė), matome kiek kitokius rezultatus. Šiuo atveju taip pat didinant aibės dydį, daugeliu atveju, absoliutinės paklaidos vidurkis mažėjo. Vienintėlis atvejis, kai didinant duomenų aibę vidurkio reikšmė suprastėjo, buvo keičiant įrašų kiekį iš 1~000 į 10~000. Kalbant apie standartinio nuokrypio reikšmes, matome, kad šis statistinis rodiklis šiuo atveju stabiliai mažėjo. 
        
        \item Atkreipus dėmesį į statistinių rodiklių pokytį keičiant duomenų aibės dydį, matyti, kad didinant taškų skaičių nuo 10~000 iki 100~000, vidurkis sumažėjo 74,8\%, o keičiant aibę iš 100~000 į 1~000~000, vidurkis sumažėjo apie 21,8\%. Tuo tarpu standartinis nuokrypis, keičiant įrašų skaičių iš 1~000 į 10~000, iš 10~000 į 100~000 bei iš 100~000 į 1~000~000, sumažėjo atitinkamai 35,7\%, 65,8\% ir 33,7\%.
        
        \begin{table}[H]
    \centering
    
    \begin{tabular}{|c|c|c|c|c|}
        \hline
        \textbf{Taškų kiekis} & \textbf{Minimumas} & \textbf{Maksimumas} & \textbf{Vidurkis} & \textbf{Standartinis nuokrypis} \\ \hline
        1 000      & 0.008972 & 9.575722 & 1.970832 & 1.910026 \\ \hline
        10 000     & 0.000362 & 5.672182 & 2.127527 & 1.227688 \\ \hline
        100 000    & 0.000038 & 2.539330 & 0.536711 & 0.420499 \\ \hline
        1 000 000  & 0.000006 & 1.809952 & 0.419722 & 0.278947 \\ \hline
    \end{tabular}
    
    \caption{Dirbtinio neuroninio tinklo rezultatai su triukšmingais duomenim}
    \label{tbl:fnn_sferos_funkc_su_triuksmu}
\end{table}


        \item Vertinant modelio tikslumo pokytį tarp švarios ir triukšmingos duomenų aibių, matyti, kad mažiausią įtaką triukšmas turėjo duomenų aibei su 1~000~000 taškų -- modelio absoliučios paklaidos vidurkis padidėjo 0,9\%, o standartinis nuokrypis -- 1,8\%. Šiek tiek didesni pokyčiai stebimi bandymuose su 1~000 įrašų duomenų aibe, kur vidurkis padidėjo 11,1\%, o standartinis nuokrypis -- 55,1\%. Didžiausi pakitimai pastebimi nagrinėjant 10~000 ir 100~000 įrašų duomenų aibes: šiose aibėse vidurkis padidėjo atitinkamai 240\% ir 146,3\%, o standartinis nuokrypis -- 138,4\% ir 144,1\%.

    \end{itemize}

\end{frame}