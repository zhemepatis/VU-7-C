\begin{frame}{Darbo tikslas ir uždaviniai}

    \begin{itemize}

        \item \textbf{Tikslas:} ištirti dirbtinių neuroninių tinklų ir $k$ artimiausių kaimynų metodo gebėjimą apdoroti triukšmingus duomenis.
        
        \item \textbf{Uždaviniai:}
        
            \begin{enumerate}
                \item Apžvelgti literatūrą apie k artimiausių kaimynų metodą, dirbtinius neuroninius tinklus ir triukšmo poveikį šių metodų veikimui;
                \item Dirbtinai sukurti duomenų aibę;
                \item Pritaikyti sukurtai duomenų aibei triukšmą;
                \item Apmokyti sukurtus $k$ artimiausių kaimynų ir dirbtinio neuroninio tinklo modelius su paruoštom duomenų aibėm;
                \item Išnagrinėti, kaip kinta modelių rezultatai didinant duomenų aibes ir į jas įterpiant triukšmą.
            \end{enumerate}

    \end{itemize}

\end{frame}