\begin{frame}{Temos aktualumas}

    \begin{itemize}

        \item Pastarajį penkmetį, dėl savo gebėjimo modeliuoti sudėtingus netiesinius ryšius ir plataus pritaikomumo įvairiose mokslo ir verslo srityse, vis didesnį populiarumą įgyja dirbtiniai neuroniniai tinklai. Su šiuo mašininio mokymosi metodu yra atliekami bandymai objektų atpažinimo paveikslėliuose, natūraliosios kalbos apdorojimo, medicininės diagnostikos ir kitose srityse. Skaičiavimo galios augimas bei didelių duomenų centrų vystymasis ir jų didėjantis prieinamumas taip pat skatina šios srities tobulėjimą.

        \item Vis dėlto realaus pasaulio taikymuose duomenys retai būna idealūs. Matavimai dažnai būna paveikti triukšmo, atsirandančio dėl jutiklių netikslumų, žmogaus klaidų, aplinkos veiksnių ar duomenų perdavimo trikdžių. Dėl to modeliai, apmokyti naudojant tokius duomenis, gali pasižymėti sumažėjusiu tikslumu ir prastesnėmis apibendrinimo savybėmis. Tai lemia, kad atsparumas triukšmingiems duomenims tampa itin svarbia dirbtinių neuroninių tinklų savybe. Dėl šios priežasites yra būtina kurti ir vertinti neorinių tinklų modelius, gebančius efektyviai apdoroti triukšmingus duomenis ir kartu išlaikyti pakankamą tikslumo lygį.

    \end{itemize}
    
\end{frame}