\begin{frame}{Temos aktualumas}

    \begin{itemize}

        \item Dirbtiniai neuroniniai tinklai pastaruoju metu vis įgija vis didesnį populiarumą dėl gebėjimo modeliuoti sudėtingus netiesinius ryšius.
        
        \item Ši mašininio mokymosi sritis yra naudojama objektų atpažinimo paveikslėliuose, natūraliosios kalbos apdorojimo, medicininės diagnostikos ir kitose srityse.
        
        \item Skaičiavimo galios augimas bei didelių duomenų centrų vystymasis ir jų didėjantis prieinamumas taip pat skatina šios srities tobulėjimą.

        \item Realaus pasaulio duomenys dažnai turi netikslumų. To priežastys gali būti įvairios -- žmogaus klaidos, prietaisų tikslumo ribos, aplinkos veiksnių poveikis ir t.t.

        \item Tyrimai rodo, kad triukšmo įterpimas modelio apmokymo metu leidžia pagerinti modelio atsparumą triukšmui ir apibendrinimo gebėjimus \cite{zur_noise_comparison} \cite{nguyen_training_more_robust_classification}.

    \end{itemize}
    
\end{frame}