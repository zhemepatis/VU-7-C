\documentclass[11pt,aspectratio=169]{beamer}

\renewcommand{\figurename}{}
\renewcommand{\tablename}{}
\renewcommand\thefigure{\arabic{figure} pav.}
\renewcommand\thetable{\arabic{table} lentelė.}

\usepackage{graphicx}
\usepackage{listings}
\usepackage[
	backend=biber,
	style=numeric,
	sorting=ynt
]{biblatex}
\usepackage{algorithm,algorithmic}
\usepackage{caption}
\usepackage{subfig}

\usepackage{biblatex}
\usepackage{hyperref}
\usepackage{fontspec}

\addbibresource{references.bib}

\usetheme{Madrid}

\AtBeginSection[]
{
   \begin{frame}<beamer>
       \frametitle{Turinys}
       \tableofcontents[currentsection]
   \end{frame}
}

\title[]{Dirbtinių neuroninių tinklų apmokymas su triukšmu}
\subtitle[]{Kursinis projektinis darbas}
\author[Gabrielė Rinkevičiūtė]{Darbą atliko: Gabrielė Rinkevičiūtė \\ Informatikos studijų programa, 4 k., 2 gr}
\date{2026-01-23}

\addtobeamertemplate{author}{}{Darbo vadovas: Asist. Dr. Linas Litvinas\par}

\setbeamertemplate{navigation symbols}{}
\titlegraphic{\includegraphics[width=2cm]{attachments/img/MIF.png}}

\usepackage{VUMIF}

\begin{document}

    \begin{frame}
        \titlepage
    \end{frame}

    \begin{frame}{Temos aktualumas}

    \begin{itemize}

        \item Dirbtiniai neuroniniai tinklai pastaruoju metu vis įgija vis didesnį populiarumą dėl gebėjimo modeliuoti sudėtingus netiesinius ryšius.
        
        \item Ši mašininio mokymosi sritis yra naudojama objektų atpažinimo paveikslėliuose, natūraliosios kalbos apdorojimo, medicininės diagnostikos ir kitose srityse.
        
        \item Skaičiavimo galios augimas bei didelių duomenų centrų vystymasis ir jų didėjantis prieinamumas taip pat skatina šios srities tobulėjimą.

        \item Realaus pasaulio duomenys dažnai turi netikslumų. To priežastys gali būti įvairios -- žmogaus klaidos, prietaisų tikslumo ribos, aplinkos veiksnių poveikis ir t.t.

        \item Tyrimai rodo, kad triukšmo įterpimas modelio apmokymo metu leidžia pagerinti modelio atsparumą triukšmui ir apibendrinimo gebėjimus \cite{zur_noise_comparison} \cite{nguyen_training_more_robust_classification}.

    \end{itemize}
    
\end{frame}
    \begin{frame}{Darbo tikslas ir uždaviniai}

    \begin{itemize}

        \item \textbf{Tikslas:} ištirti dirbtinių neuroninių tinklų ir $k$ artimiausių kaimynų metodo gebėjimą apdoroti triukšmingus duomenis.
        
        \item \textbf{Uždaviniai:}
        
            \begin{enumerate}
                \item Apžvelgti literatūrą apie k artimiausių kaimynų metodą, dirbtinius neuroninius tinklus ir triukšmo poveikį šių metodų veikimui;
                \item Dirbtinai sukurti duomenų aibę;
                \item Pritaikyti sukurtai duomenų aibei triukšmą;
                \item Apmokyti sukurtus $k$ artimiausių kaimynų ir dirbtinio neuroninio tinklo modelius su paruoštom duomenų aibėm;
                \item Išnagrinėti, kaip kinta modelių rezultatai didinant duomenų aibes ir į jas įterpiant triukšmą.
            \end{enumerate}

    \end{itemize}

\end{frame}

    \section{Duomenų aibė}

        \begin{frame}{Duomenų aibė}

    \begin{itemize}
        
        \item Bandymams atlikti buvo generuojami keturmačiai duomenys, kurių kiekviena komponentė buvo parenkama iš intervalo $[-5; 5]$. Duomenys buvo generuoti naudojant sferos (\ref{eq:sferos_etalono_funkc} lygtis) funkciją. 
        
        \item Modeliai buvo pritaikyti skirtingo dydžio aibėms. Dirbtinis neuroninis tinklas buvo pritaikytas su 1~000, 10~000, 100~000, 1~000~000 dydžio duomenų aibėmis, tuo tarpu $k$ artimiausių kaimynų metodas dar papildomai buvo išbandytas su aibe, turinčia 10~000~000 įrašų. 
        
        \item Triukšmo pridėjimui į duomenų aibę buvo pasitelktas Gauso paskirsymo funkcijos metodas. Paklaidos buvo pridedamos prie sugeneruotų etalono funkcijų reikšmių. Triukšmas buvo generuojamos su parametrais $\mu = 0$, $\sigma = 0.5$.

        \item Prieš apmokant dirbtinio neuroninio tinklos modelį duomenys buvo normalizuojami min-max metodu. Perėjus prie modelio statistinių rodiklių įvertinimo duomenys buvo grąžinami į pradinį duomenų mąstelį. Tad atsižvelgus į tai, kad duomenys buvo generuoti sferos funkcija prognozės absoliutinė paklaida įeina į intervalą $[0; 100].$

    \end{itemize}
    
\end{frame}
    
    \section{Modeliai}
    
    \section{Tyrimo rezultatai}

        \begin{frame}[allowframebreaks]{Tyrimo rezultatai}

    \begin{itemize}

        \item Nagrinėjant modelio rezultatus su švariais duomenim (\ref{tbl:fnn_sferos_funkc_svarus} lentelė) matome, kad, didžiąja dalimi, didinant taškų kiekį, modelio absoliutinės paklaidos vidurkis ir standartinis nuokrypis mažėjo. Išsiskiria tik atvejis su 1~000~000 taškų aibe - modelis grąžino blogesnius rezultatus lyginant su rezultatais, gautais naudojant 100~000 taškų aibę. Žvelgiant į standartinio nuokrypio reikšmes, matome analogišką dėsningumą.

        \item Lyginant modelio rodiklius skirtingam taškų kiekiui, matyti, kad abiem gerėjimo atvejais absoliučios paklaidos vidurkis sumažėjo atitinkamai 64,7\% ir 65,1\%. Tuo tarpu standartinis nuokrypis, didinant duomenų aibės dydį nuo 1~000 iki 10~000, sumažėjo 58,1\%, o nuo 1~000 iki 100~000 -- 66,6\%.        
        
        \begin{table}[H]
    \centering
    
    \begin{tabular}{|c|c|c|c|c|}
        \hline
        \textbf{Taškų kiekis} & \textbf{Minimumas} & \textbf{Maksimumas} & \textbf{Vidurkis} & \textbf{Standartinis nuokrypis} \\ \hline
        1 000      & 0.002857 & 6.881401 & 1.773480 & 1.231157 \\ \hline
        10 000     & 0.001137 & 4.690506 & 0.625588 & 0.514914 \\ \hline
        100 000    & 0.000001 & 1.170944 & 0.217896 & 0.172228 \\ \hline
        1 000 000  & 0.000004 & 1.631981 & 0.416016 & 0.274041 \\ \hline
    \end{tabular}
    
    \caption{Dirbtinio neuroninio tinklo rezultatai su švariais sferos funkcijos generuotais duomenis}
    \label{tbl:fnn_sferos_funkc_svarus}
\end{table}


        \item Pažvelgus į modelio rezultatus su triukšmingais duomenim (\ref{tbl:fnn_sferos_funkc_su_triuksmu} lentelė), matome kiek kitokius rezultatus. Šiuo atveju taip pat didinant aibės dydį, daugeliu atveju, absoliutinės paklaidos vidurkis mažėjo. Vienintėlis atvejis, kai didinant duomenų aibę vidurkio reikšmė suprastėjo, buvo keičiant įrašų kiekį iš 1~000 į 10~000. Kalbant apie standartinio nuokrypio reikšmes, matome, kad šis statistinis rodiklis šiuo atveju stabiliai mažėjo. 
        
        \item Atkreipus dėmesį į statistinių rodiklių pokytį keičiant duomenų aibės dydį, matyti, kad didinant taškų skaičių nuo 10~000 iki 100~000, vidurkis sumažėjo 74,8\%, o keičiant aibę iš 100~000 į 1~000~000, vidurkis sumažėjo apie 21,8\%. Tuo tarpu standartinis nuokrypis, keičiant įrašų skaičių iš 1~000 į 10~000, iš 10~000 į 100~000 bei iš 100~000 į 1~000~000, sumažėjo atitinkamai 35,7\%, 65,8\% ir 33,7\%.
        
        \begin{table}[H]
    \centering
    
    \begin{tabular}{|c|c|c|c|c|}
        \hline
        \textbf{Taškų kiekis} & \textbf{Minimumas} & \textbf{Maksimumas} & \textbf{Vidurkis} & \textbf{Standartinis nuokrypis} \\ \hline
        1 000      & 0.008972 & 9.575722 & 1.970832 & 1.910026 \\ \hline
        10 000     & 0.000362 & 5.672182 & 2.127527 & 1.227688 \\ \hline
        100 000    & 0.000038 & 2.539330 & 0.536711 & 0.420499 \\ \hline
        1 000 000  & 0.000006 & 1.809952 & 0.419722 & 0.278947 \\ \hline
    \end{tabular}
    
    \caption{Dirbtinio neuroninio tinklo rezultatai su triukšmingais duomenim}
    \label{tbl:fnn_sferos_funkc_su_triuksmu}
\end{table}


        \item Vertinant modelio tikslumo pokytį tarp švarios ir triukšmingos duomenų aibių, matyti, kad mažiausią įtaką triukšmas turėjo duomenų aibei su 1~000~000 taškų -- modelio absoliučios paklaidos vidurkis padidėjo 0,9\%, o standartinis nuokrypis -- 1,8\%. Šiek tiek didesni pokyčiai stebimi bandymuose su 1~000 įrašų duomenų aibe, kur vidurkis padidėjo 11,1\%, o standartinis nuokrypis -- 55,1\%. Didžiausi pakitimai pastebimi nagrinėjant 10~000 ir 100~000 įrašų duomenų aibes: šiose aibėse vidurkis padidėjo atitinkamai 240\% ir 146,3\%, o standartinis nuokrypis -- 138,4\% ir 144,1\%.

    \end{itemize}

\end{frame}
    
    \section{Išvados ir pasiūlymai tolimesniem darbam}
        
        \section{Išvados ir pasiūlymai tolimesniem darbam}

    \begin{frame}[allowframebreaks]{Išvados ir pasiūlymai tolimesniems darbams}

        \begin{itemize}
            
            \item Analizuojant tiek švarias, tiek triukšmingas duomenų aibes, buvo padarytos tos pačios išvados:

            \begin{itemize}
                \item dirbtinis neuroninis tinklas visais nagrinėtais atvejais pasižymėjo mažesnėmis absoliutinės paklaidos vidurkio ir standartinio nuokrypio reikšmėmis;
                \item $k$ artimiausių kaimynų metodo atveju nustatyta, kad didėjant mokymo duomenų aibės apimčiai, modelio tikslumas didėjo nuosekliai ir panašiu tempu;
                \item dirbtinio neuroninio tinklo atveju tikslumo rodiklių pokyčiai buvo mažiau stabilūs, o kai kuriais atvejais stebėtas ir jų pablogėjimas.
            \end{itemize}

            \break

            \item Lyginant, kaip įterptas triukšmas paveikė modelių tikslumo rodiklius, buvo nustatytos šios tendencijos:

            \begin{itemize}
                \item didėjant duomenų aibės apimčiai, $k$ artimiausių kaimynų metodui būdingas vis didesnis tikslumo rodiklių skirtumas tarp švarių ir triukšmingų duomenų aibių;
                \item dirbtinio neuroninio tinklo atveju buvo stebėtas reikšmingai didesnis tikslumo rodiklių pablogėjimas, lyginant su $k$ artimiausių kaimynų metodo rezultatais.
            \end{itemize}

            \break

            \item Siūlymai ateities darbam:
            
            \begin{itemize}
                \item atlikti išsamesnę triukšmo poveikio analizę, keičiant triukšmo intensyvumo parametrus ir triukšmo tipą;
                \item pritaikyti modelius kitokio pobūdžio aibėms, siekiant įvertinti gautų rezultatų bendrumą;
                \item išnagrinėti triukšmo poveikį įvarioms dirbtinių neuroninių tinklų architektūroms;
                \item ištirti, kaip triukšmo poveikis $k$ artimiausių kaimynų modeliui kinta priklausomai nuo parametro $k$ pasirinkimo.
            \end{itemize}
            
        \end{itemize}
        
    \end{frame}

\end{document}
